\begin{chapterpage}{Inference for categorical data}
  \chaptertitle{Inference for categorical data}
  \label{inferenceForCategoricalData}
  \label{ch_inference_for_props}
  \chaptersection{singleProportion}
  \chaptersection{differenceOfTwoProportions}
  \chaptersection{oneWayChiSquare}
  \chaptersection{twoWayTablesAndChiSquare}
\end{chapterpage}
\renewcommand{\chapterfolder}{ch_inference_for_props}

\chapterintro{In this chapter,
  we apply the methods and ideas from
  Chapter~\ref{ch_foundations_for_inf}
  in several contexts for categorical data.
  We'll start by revisiting what we learned for a single
  proportion, where the normal distribution can be used
  to model the uncertainty in the sample proportion.
  Next, we apply these same ideas to analyze the difference
  of two proportions using the normal model.
  Later in the chapter, we apply inference techniques
  to contingency tables;
  while we will use a different
  distribution in this context, the core
  ideas of hypothesis testing remain the same.}


%__________________
\section{Inference for a single proportion}
\label{singleProportion}

We encountered inference methods for a single proportion
in Chapter~\ref{ch_foundations_for_inf},
exploring point estimates, confidence intervals,
and hypothesis tests.
In this section, we'll do a review of these topics
and also explore how to perform sample size calculations
for data collection purposes in the context of a single
proportion.


\subsection{Identifying when the sample proportion is nearly normal}

A sample proportion $\hat{p}$ will tend to be well-modeled
using a normal distribution when the observations in the
sample are independent and the sample size is sufficiently
large.


%A sample proportion can be described as a sample mean. If we represent each ``success'' as a 1 and each ``failure'' as a 0, then the sample proportion is the mean of these numerical outcomes:
%\begin{eqnarray*}
%\hat{p} = \frac{\ 0 + 1 + 1 + \cdots + 0\ }{1042} = 0.82
%\end{eqnarray*}
%The distribution of $\hat{p}$ is nearly normal when the distribution of 0's and 1's is not too strongly skewed for the sample size. The most common guideline for sample size and skew when working with proportions is to ensure that we expect to observe a minimum number of successes (1's) and failures (0's), typically at least 10 of each. The labels \term{success} and \term{failure} need not mean something positive or negative. These terms are just convenient words that are frequently used when discussing proportions.

\begin{onebox}{Sampling distribution of $\mathbf{\hat{p}}$}
  The sampling distribution for $\hat{p}$, taken from
  a sample of size $n$ from a population with a true
  proportion $p$, is nearly normal when:
  \begin{enumerate}
  \setlength{\itemsep}{0mm}
  \item The sample observations are independent,
      e.g. are from a simple random sample.
  \item We expected to see at least 10 successes and
      10 failures in the sample, i.e. $np\geq10$ and
      $n(1-p)\geq10$.
      This is called the \term{success-failure condition}.
  \end{enumerate}
  When these conditions are met, then the sampling
  distribution of $\hat{p}$ is nearly normal with mean
  $p$ and standard error
  \index{standard error (SE)!single proportion}%
  $SE_{\hat{p}} = \sqrt{\frac{\ p(1-p)\ }{n}}$.
\end{onebox}

Typically we don't know the true proportion, $p$,
so we substitute some value to check conditions
and estimate the standard error.
For confidence intervals, usually the sample proportion
$\hat{p}$ is used to check the success-failure condition
and compute the standard error.
For hypothesis tests, typically the null value --
that is, the proportion claimed in the null hypothesis --
is used in place of $p$.


\subsection{Confidence intervals for a proportion}
\label{confIntForPropSection}

\index{point estimate!single proportion}

Often times we want to understand what a range of
plausible values for the parameter, $p$.
When this is our goal, a confidence interval is useful,
which for a proportion usually takes the form of
\begin{align*}
\hat{p} \pm z^{\star} \times SE_{\hat{p}}
\end{align*}
%where $\hat{p}$ is the point estimate based on
%a simple random sample.
A core requirement is that $\hat{p}$ can be reasonably
modeled with a normal distribution, and we check this
requirement using conditions.

%In this first example, we'll use the point estimate
%$\hat{p}$ to check whether it is reasonable to model
%$\hat{p}$ using a normal distribution,
%since we will be computing a confidence interval.

\index{data!Payday regulation poll|(}

\newcommand{\paydayN}{826}
\newcommand{\paydayNHalf}{413}
\newcommand{\paydayRegPerc}{70\%}
\newcommand{\paydayRegProp}{0.70}
\newcommand{\paydayRegSE}{0.016}
\newcommand{\paydayRegSEPerc}{1.6\%}
\newcommand{\paydayRegLower}{0.669}
\newcommand{\paydayRegUpper}{0.731}
\newcommand{\paydayRegLowerPerc}{66.9\%}
\newcommand{\paydayRegUpperPerc}{73.1\%}
% https://www.pewtrusts.org/-/media/assets/2017/04/payday-loan-customers-want-more-protections-methodology.pdf

\begin{examplewrap}
\begin{nexample}{A simple random sample of \paydayN{}
    payday loan borrowers was surveyed to better
    understand their interests around regulation and costs.
    \paydayRegPerc{} of the responses supported new
    regulations on payday lenders.
    Is it reasonable to model $\hat{p} = \paydayRegProp{}$
    using a normal distribution?}
  This is a random sample, so the observations are
  independent and representative of the population of interest.

  We also must check the success-failure condition.
  In the example, we are defining a \emph{success}
  as a respondent who supported new regulations
  and a \emph{failure} as someone who did not.
  We check this condition using $\hat{p}$ in place
  of $p$ when computing a confidence interval:
  \begin{align*}
  \text{Successes: }
      n p & \approx \paydayN{} * \paydayRegProp{}
      = 578
  &\text{Failures: }
      n (1 - p) & \approx \paydayN{} * (1 - \paydayRegProp{})
      = 248
  \end{align*}
  Since both values are at least 10, we can use the normal
  distribution to model $\hat{p}$.
\end{nexample}
\end{examplewrap}




\begin{exercisewrap}
\begin{nexercise} \label{seOfPropOfPDBorrowersSupportReg}
Estimate the standard error of $\hat{p} = \paydayRegProp{}$.
Because $p$ is unknown and the standard error is for
a confidence interval, use $\hat{p}$ in place of $p$
in the formula.\footnotemark
\end{nexercise}
\end{exercisewrap}
\footnotetext{$SE = \sqrt{\frac{p(1-p)}{n}} \approx
    \sqrt{\frac{\paydayRegProp{} (1 - \paydayRegProp{})}
        {\paydayN{}}} = \paydayRegSE{}$.}

\begin{examplewrap}
\begin{nexample}{Construct a 95\% confidence interval for $p$,
    the proportion of payday borrowers who support increased
    regulation for payday lenders.}
  Using the standard error $SE = \paydayRegSE{}$ from
  Guided Practice~\ref{seOfPropOfPDBorrowersSupportReg},
  the point estimate \paydayRegProp{},
  and $z^{\star} = 1.96$ for a 95\% confidence interval,
  the confidence interval is
  \begin{eqnarray*}
  \text{point estimate} \ \pm\ z^{\star} \times SE
      \quad\to\quad
      \paydayRegProp{} \ \pm\ 1.96 \times \paydayRegSE{}
      \quad\to\quad
      (\paydayRegLower{}, \paydayRegUpper{})
  \end{eqnarray*}
  We are 95\% confident that the true proportion of
  payday borrowers who supported regulation at the time
  of the poll was between \paydayRegLower{} and
  \paydayRegUpper{}.
\end{nexample}
\end{examplewrap}

\onepropconfintsummary{}
%\begin{onebox}{Constructing a confidence interval for a proportion}
%  There are three steps to constructing a confidence
%  interval for $p$.
%  \begin{itemize}
%  \setlength{\itemsep}{0mm}
%  \item Check independence and the success-failure condition
%      using $\hat{p}$.
%      If the conditions are met, the sampling distribution
%      of $\hat{p}$ may be well-approximated by the normal model.
%  \item Construct the standard error using $\hat{p}$
%      in place of $p$ in the standard error formula.
%  \item Apply the general confidence interval formula.
%  \end{itemize}
%\end{onebox}

\noindent%
For additional one-proportion confidence interval examples,
see Section~\ref{confidenceIntervals}.


\subsection{Hypothesis testing for a proportion}
\label{htForPropSection}

\newcommand{\paydayCCPerc}{51\%}
\newcommand{\paydayCCProp}{0.51}
\newcommand{\paydayCCSE}{0.017}
\newcommand{\paydayCCSEPerc}{1.7\%}
\newcommand{\paydayCCZ}{0.59}
\newcommand{\paydayCCOneTail}{0.2776}
\newcommand{\paydayCCPvalue}{0.5552}

One possible regulation for payday lenders is that they
would be required to do a credit check and evaluate debt
payments against the borrower's finances.
We would like to know: would borrowers support this form
of regulation?


\begin{exercisewrap}
\begin{nexercise}
\label{paydayCC_hypotheses_gp}%
Set up hypotheses to evaluate whether borrowers
have a majority support or majority opposition for this
type of regulation.\footnotemark
\end{nexercise}
\end{exercisewrap}
\footnotetext{$H_0$: $p = 0.50$. $H_A$: $p \neq 0.50$.}

To apply the normal distribution framework in the context
of a hypothesis test for a proportion, the independence
and success-failure conditions must be satisfied.
In a hypothesis test, the success-failure condition is
checked using the null proportion:
we verify $np_0$ and $n(1-p_0)$ are at least 10,
where $p_0$ is the null value.

\begin{exercisewrap}
\begin{nexercise}
\label{paydayCC_conditions_gp}%
We consider another question asked
of the payday loan borrowers:
Do you support a regulation that would
require lenders to pull your credit report
and evaluate your debt payments?
Of the \paydayN{} borrowers, \paydayCCPerc{}
said they supported such a regulation.
Is it reasonable to model $\hat{p} = \paydayCCProp{}$
for a hypothesis test here?\footnotemark
\end{nexercise}
\end{exercisewrap}
\footnotetext{Independence holds since the poll
    is based on a random sample.
    The success failure also holds, which we check
    using the null proportion, $p_0 = 0.5$ from the
    null hypothesis:
    $np \approx \paydayN{} \times 0.5 = \paydayNHalf$,
    $n(1 - p) \approx \paydayN{} \times 0.5 = \paydayNHalf$.}
    
\begin{examplewrap}
\begin{nexample}{Using the hypotheses and data from
    Guided Practice~\ref{paydayCC_hypotheses_gp}
    and~\ref{paydayCC_conditions_gp},
    evaluate whether the poll provides convincing evidence
    that a majority of payday loan borrowers support
    a new regulation that would
    require lenders to pull credit reports
    and evaluate debt payments.}
  With hypotheses already set up and conditions checked,
  we can move onto calculations.
  The standard error in the context of a one-proportion
  hypothesis test is computed using the null value, $p_0$:
  \begin{align*}
  SE = \sqrt{\frac{p_0 (1 - p_0)}{n}}
      = \sqrt{\frac{0.5 (1 - 0.5)}{\paydayN{}}}
      = \paydayCCSE{}
  \end{align*}
  A picture of the normal model is shown below
  with the p-value represented by the shaded region.
  \begin{center}
  \Figure{0.5}{paydayCC_norm_pvalue}
  \end{center}
  Based on the normal model, the test statistic can be
  computed as the Z-score of the point estimate:
  \begin{align*}
  Z = \frac{\text{point estimate} - \text{null value}}{SE}
      = \frac{\paydayCCProp{} - 0.50}{\paydayCCSE{}}
      = \paydayCCZ{}
  \end{align*}
  The single tail area is \paydayCCOneTail{}, and the p-value,
  represented by both tail areas together, is \paydayCCPvalue{}.
  Because the p-value is larger than 0.05,
  we do not reject $H_0$.
  The poll does not provide convincing evidence that
  a majority of payday loan borrowers support or oppose
  regulations around credit checks and evaluation of
  debt payments.
\end{nexample}
\end{examplewrap}

\oneprophtsummary{}

%\begin{onebox}{Hypothesis test for a proportion}
%Set up hypotheses and verify the conditions using the null value, $p_0$, to ensure $\hat{p}$ is nearly normal under $H_0$. If the conditions hold, construct the standard error, again using $p_0$, and show the p-value in a drawing. Lastly, compute the p-value and evaluate the hypotheses.
%\end{onebox}

\noindent%
For additional one-proportion hypothesis test examples,
see Section~\ref{hypothesisTesting}.

\index{data!Payday regulation poll|)}

\CalculatorVideos{confidence intervals and hypothesis tests for a single proportion}


\D{\newpage}

\subsection{When one or more conditions aren't met}

We've spent a lot of time discussing conditions for when
$\hat{p}$ can be reasonably modeled by a normal distribution.
What happens when the success-failure condition fails?
What about when the independence condition fails?
In either case, the general ideas of confidence intervals
and hypothesis tests remain the same, but the strategy
or technique used to generate the interval or p-value
would change.

When the success-failure condition isn't met
for a hypothesis test, we can simulate the null distribution
of $\hat{p}$ using the null value, $p_0$.
The simulation concept is similar to the ideas used
in the malaria case study presented in
Section~\ref{caseStudyMalariaVaccine},
and an online section outlines this strategy:
\begin{center}
\oiRedirect{stat_sim_prop_ht}
    {www.openintro.org/r?go=stat\_sim\_prop\_ht}
\end{center}
\Comment{Port over the deleted OS3 section to be an online extra.}
For a confidence interval when the success-failure condition
isn't met, we can use what's called
the \term{Clopper-Pearson interval},
where the details of this method live an internet search away,
even if those details are beyond the scope of this book.

The independence condition is a more nuanced requirement.
When it isn't met, it is important to understand how and why
it isn't met.
For example, if we took a cluster sample
(see Section~\ref{section_obs_data_sampling}),
suitable statistical methods are available but would
be beyond the scope of even most second or third courses
in statistics.
On the other hand, we'd be stretched to find any method
that we could confidently apply to correct the inherent biases
of data from a convenience sample.

While this book is scoped to well-constrained statistical
problems, do remember that this is just the first
book in what is a large library of statistical methods that
are suitable for a very wide range of data and contexts.


\D{\newpage}

\subsection{Choosing a sample size when estimating a proportion}

\index{margin of error|(}

When collecting data, we choose a sample size suitable for the purpose of the study. Often times this means choosing a sample size large enough that the \term{margin of error} -- which is the part we add and subtract from the point estimate in a confidence interval -- is sufficiently small that the sample is useful. More explicitly, our task is to find a sample size $n$ so that the sample proportion is within some margin of error $m$ of the actual proportion with a certain level of confidence.

% For example, the margin of error for a point estimate using 95\% confidence can be written as $1.96\times SE$. We set up a general equation to represent the problem:
%\begin{align*}
%ME = z^{\star} \times SE \leq m
%\end{align*}
%where $ME$ represented the actual margin of error and $z^{\star}$ was chosen to correspond to the confidence level. The standard error formula is specified to correspond to the particular setting. For instance, in the case of means, the standard error was given as $\sigma / \sqrt{n}$. In the case of a single proportion, we use $\sqrt{p(1-p) / n\ }$ for the standard error.

\index{data!Student football stadium|(}

\begin{examplewrap}
\begin{nexample}{A university newspaper is conducting a survey to determine what fraction of students support a \$200 per year increase in fees to pay for a new football stadium. How big of a sample is required to ensure the margin of error is smaller than 0.04 using a 95\% confidence level?}
The margin of error for a sample proportion is
\begin{align*}
z^{\star} \sqrt{\frac{p (1 - p)}{n}}
\end{align*}
Our goal is to find the smallest sample size $n$ so that this margin of error is smaller than $m=0.04$. For a 95\% confidence level, the value $z^{\star}$ corresponds to 1.96:
\begin{align*}
1.96\times \sqrt{\frac{p(1-p)}{n}} \ < \ 0.04
\end{align*}
There are two unknowns in the equation: $p$ and $n$. If we have an estimate of $p$, perhaps from a similar survey, we could enter in that value and solve for $n$. If we have no such estimate, we must use some other value for $p$. It turns out that the margin of error is largest when $p$ is 0.5, so we typically use this \emph{worst case value} if no estimate of the proportion is available:
\begin{align*}
	1.96\times \sqrt{\frac{0.5(1-0.5)}{n}} &\ < \ 0.04 \\
	1.96^2\times \frac{0.5(1-0.5)}{n} &\ < \ 0.04^2 \\
	1.96^2\times \frac{0.5(1-0.5)}{0.04^2} &\ < \ n \\
	600.25 &\ < \  n
\end{align*}
We would need over 600.25 participants, which means we need 601 participants or more, to ensure the sample proportion is within 0.04 of the true proportion with 95\% confidence.
\end{nexample}
\end{examplewrap}

\index{data!Student football stadium|)}

When an estimate of the proportion is available, we use it in place of the worst case proportion value,~0.5.

\index{data!Tire failure rate|(}

\begin{exercisewrap}
\begin{nexercise}
\label{tire_failure_rate_3_samp_size_calc}%
A manager is about to oversee the mass
production of a new tire model in her factory,
and she would like to estimate what proportion of
these tires will be rejected through quality control.
The quality control team has monitored the last three
tire models produced by the factory,
failing 1.7\% of tires in the first model,
6.2\% of the second model,
and 1.3\% of the third model.
The manager would like to examine enough tires
to estimate the failure rate of the new tire model
to within about 1\% with a 90\% confidence level.
There are three different failure rates to choose from.
Perform the sample size computation for each separately,
and identify three sample sizes to consider.\footnotemark
\end{nexercise}
\end{exercisewrap}
\footnotetext{For a 90\% confidence interval, $z^{\star} = 1.65$,
  and since an estimate of the proportion 0.017 is available,
  we'll use it in the margin of error formula:
  \begin{align*}
  1.65\times \sqrt{\frac{0.017(1-0.017)}{n}} &\ < \ 0.01
    \qquad\to\qquad
      \frac{0.017(1-0.017)}{n} \ < \ 
          \left(\frac{0.01}{1.65}\right)^2
    \qquad\to\qquad
      454.96 \ < \ n
  \end{align*}
  For sample size calculations, we always round up,
  so the first tire model suggests 455 tires would
  be sufficient.

  A similar computation can be accomplished using 0.062
  and 0.013 for $p$, and you should verify that using these
  proportions results in minimum sample sizes of 1584
  and~350 tires, respectively.}

\begin{examplewrap}
\begin{nexample}{The sample sizes vary widely in
    Guided Practice~\ref{tire_failure_rate_3_samp_size_calc}.
    Which of the three would you suggest using?
    What would influence your choice?}
  We could examine which of the old models is most
  like the new model, then choose the corresponding sample
  size.
  Or if two of the previous estimates are based on small
  samples while the other is based on a larger sample,
  we should consider the value corresponding to the larger
  sample.
  There are also other reasonable approaches.

  Also observe that the success-failure
  condition would need to be checked in the final sample.
  For instance, if we sampled $n = 1584$ tires and found
  a failure rate of 0.5\%, the normal approximation would
  not be reasonable, and we would require more advanced
  statistical methods for creating the confidence interval.
\end{nexample}
\end{examplewrap}

\index{data!Tire failure rate|)}
\index{data!Payday regulation poll|(}

\begin{exercisewrap}
\begin{nexercise}
Suppose we want to continually track the support
of payday borrowers for regulation on lenders,
where we would conduct a new poll every month.
Running such frequent polls is expensive, so we decide
a wider margin of error of 5\% for each individual survey
would be acceptable.
Based on the original sample of borrowers where
\paydayRegPerc{} supported some form of regulation,
how big should our monthly sample be for a margin
of error of 0.04 with 95\% confidence?\footnotemark
\end{nexercise}
\end{exercisewrap}
\footnotetext{We complete the same computations as before,
   except now we use $\paydayRegProp{}$ instead of $0.5$
   for $p$:
   \begin{align*}
   1.96\times \sqrt{\frac{p(1-p)}{n}}
       \approx 1.96\times
           \sqrt{\frac{\paydayRegProp{}(1-\paydayRegProp{})}
               {n}}
       &\leq 0.05
     \qquad\to\qquad
       n \geq 322.7
  \end{align*}
  A sample size of 323 or more would be reasonable.
  (Reminder: always round up for sample size calculations!)
  Given that we plan to track this poll over time,
  we also may want to periodically repeat these calculations
  to ensure that we're being thoughtful in our sample
  size recommendations.}

\index{data!Payday regulation poll|)}
\index{margin of error|)}




%__________________
\section{Difference of two proportions}
\label{differenceOfTwoProportions}

We would like to make conclusions about the difference
in two population proportions: \mbox{$p_1 - p_2$}.
We consider three examples.
In the first, we compare the utility of a blood thinner
for heart attack patients.
In the second application, we examine the efficacy of
mammograms in reducing deaths from breast cancer.
In the last example, a quadcopter company weighs whether
to switch to a higher quality manufacturer of rotor blades.

In our investigations, we first identify a reasonable
point estimate of $p_1 - p_2$ based on the sample.
You may have already guessed its form:
$\hat{p}_1 - \hat{p}_2$.
\index{point estimate!difference of proportions}
Next, we'll use the same process as
what we followed in the
single-proportion context:
we verify that the point estimate
follows a normal distribution,
compute the estimate's standard error, and
apply our inferential framework.


\subsection{Sample distribution of the difference of two proportions}

Like with $\hat{p}$, the difference of two sample proportions
$\hat{p}_1 - \hat{p}_2$ follows a normal distribution when
certain conditions are met.
%We must check two conditions before modeling
%$\hat{p}_1 - \hat{p}_2$ using a normal distribution.
First, the sampling distribution for each sample
proportion must be nearly normal, and secondly,
the samples must be independent.
Under these two conditions, the sampling distribution
of $\hat{p}_1 - \hat{p}_2$ may be well approximated
using a normal distribution.

\begin{onebox}{Conditions for the
    sampling distribution of $\hat{p}_1 - \hat{p}_2$
    to be normal}
  The difference $\hat{p}_1 - \hat{p}_2$ can be modeled
  using a normal distribution when
  \begin{itemize}
  \setlength{\itemsep}{0mm}
  \item the two samples are independent of each other, and
  \item each proportion separately follows a normal model.
  \end{itemize}
  When these conditions are satisfied,
  the standard error of $\hat{p}_1 - \hat{p}_2$ is
  \index{standard error (SE)!difference in proportions}
  \begin{eqnarray*}
  SE_{\hat{p}_1 - \hat{p}_2}
    = \sqrt{SE_{\hat{p}_1}^2 + SE_{\hat{p}_2}^2}
    = \sqrt{\frac{p_1(1-p_1)}{n_1} + \frac{p_2(1-p_2)}{n_2}}
  \label{seForDiffOfProp}
  \end{eqnarray*}
  where $p_1$ and $p_2$ represent the population proportions,
  and $n_1$ and $n_2$ represent the sample~sizes.
\end{onebox}

In the one-proportion context, we used the point estimate
and standard error to create confidence intervals or conduct
hypothesis tests.
We can do the same in two-proportion applications.

%For the difference in two means, the standard error formula took the following form:
%\begin{eqnarray*}
%SE_{\bar{x}_{1} - \bar{x}_{2}} = \sqrt{SE_{\bar{x}_1}^2 + SE_{\bar{x}_2}^2}
%\end{eqnarray*}
%The standard error for the difference in two proportions takes a similar form. The reasons behind this similarity are rooted in the probability theory of Section~\ref{randomVariablesSection}, which is described for this context in Guided Practice~\vref{derivingSEForDiffOfTwoMeansExercise}.


\subsection{Confidence intervals for $\mathbf{p_1 - p_2}$}

\index{data!CPR and blood thinner|(}

%In the setting of confidence intervals for a difference
%of two proportions, the two sample proportions are used
%to verify the success-failure condition and also compute
%the standard error, just as was the case with a single
%proportion.
\noindent%
We can use
the generic formula of a confidence interval, just as it did in the one-proportion case:
\begin{align*}
\text{point estimate} \pm z^{\star} \times SE
\end{align*}
We can follow the same Prepare, Check, Calculate, Conclude
steps for
computing a confidence interval
or
completing a hypothesis test
for a difference of two proportions.
The details change a little,
but the general approach remain the same.
Think about these steps when you apply statistical methods.

\begin{examplewrap}
\begin{nexample}{We consider an experiment for patients
    who underwent CPR for a heart attack and were
    subsequently admitted to a
    hospital.
    These patients were randomly divided into a treatment
    group where they received a blood thinner or the control
    group where they did not receive a blood thinner.
    The outcome variable of interest was whether the
    patients survived for at least 24 hours.
    The results are shown in
    Figure~\ref{resultsForCPRStudyInSmallSampleSection}.
    Check whether we can model the difference in
    sample proportions using the normal distribution.}

  %Before proceeding, the conditions must be checked.
  We check if each proportion separately can be
  modeled using a normal distribution:
%  \begin{itemize}
%  \setlength{\itemsep}{0mm}
%  \item
      Because the patients were randomized to their groups,
      the observations in the two samples are independent,
      both within and between groups.
%  \item
      The success-failure condition also holds for each group,
      where we have at least 10 successes and 10 failures in
      each experiment arm.
%  \end{itemize}
  With these conditions satisfied,
  the difference in sample proportions can be
  reasonably modeled using a normal distribution.
\end{nexample}
\end{examplewrap}

\begin{figure}[ht]
\centering
\begin{tabular}{lccccc}
\hline
			&& Survived 	& Died 	&& Total \\
\hline
Control		&& 11		& 39		&& 50 \\
Treatment		&& 14		& 26		&& 40 \\
\hline
Total			&& 25		& 65		&& 90 \\
\hline
\end{tabular}
\caption{Results for the CPR study.
    Patients in the treatment group were given
    a blood thinner, and patients in the control
    group were not.}
\label{resultsForCPRStudyInSmallSampleSection}
\end{figure}

\begin{examplewrap}
\begin{nexample}{
    Create and interpret a 90\% confidence interval of the
    difference for the survival rates in the CPR study.}

  We'll use $p_t$ for the survival
  rate in the treatment group and $p_c$ for the control
  group:
  \begin{align*}
  \hat{p}_{t} - \hat{p}_{c}
    = \frac{14}{40} - \frac{11}{50}
    = 0.35 - 0.22
    = 0.13
  \end{align*}
  We use the standard error formula provided on
  page~\pageref{seForDiffOfProp}.
  As with the one-sample proportion case,
  we use our best estimates of each proportion in the formula:
  \begin{align*}
  SE \approx \sqrt{\frac{0.35 (1 - 0.35)}{40} +
      \frac{0.22 (1 - 0.22)}{50}}
    = 0.095
  \end{align*}
  For a 90\% confidence interval, we use $z^{\star} = 1.65$:
  \begin{align*}
  \text{point estimate} \ \pm\ z^{\star} \times SE
    \quad \to \quad 0.13 \ \pm\ 1.65 \times  0.095
    \quad \to \quad (-0.027, 0.287)
  \end{align*}
  We are 90\% confident that that blood thinners have
  a difference of -2.7\% to +28.7\% percentage point
  impact on survival rate for patients like those in
  the study.
  Because 0\% is contained in the interval,
  we do not have enough information to say
  with confidence whether blood thinners help or harm
  heart attack patients who have been admitted after
  they have undergone CPR.
\end{nexample}
\end{examplewrap}

\index{data!CPR and blood thinner|)}

%\begin{onebox}{Confidence interval for a difference
%    of two proportions}
%  Once you've determined a confidence interval for the
%  difference of two proportions would be helpful for an
%  application, there are four steps to constructing the interval:
%  \begin{description}
%  \item[Prepare.]
%      Identify the sample proportions and sample sizes
%      for each of the two groups,
%      determine what confidence level you wish to use.
%  \item[Check.]
%      Verify the conditions to ensure each sample
%      proportion is nearly normal.
%      The success-failure condition should be checked
%      for each group.
%  \item[Calculate.]
%      If the conditions hold, compute $SE$,
%      find $z^{\star}$, and construct the interval.
%  \item[Conclude.]
%      Interpret the confidence interval in the context
%      of the problem.
%  \end{description}
%\end{onebox}

\begin{exercisewrap}
\begin{nexercise}
A 5-year experiment
was conducted to evaluate the effectiveness
of fish oils on reducing cardiovascular events,
where each subject was randomized into one of two
treatment groups.
We'll consider heart attack outcomes in these patients:
\begin{center}
\begin{tabular}{l ccc}
  \hline
  & heart\_\hspace{0.3mm}attack &
      no\_\hspace{0.3mm}event & Total \\
  \hline
  fish\_\hspace{0.3mm}oil & 145 & 12788 & 12933 \\
  placebo & 200 & 12738 & 12938 \\
  \hline
\end{tabular}
\end{center}
% library(openintro); library(xtable); xtable(fish_oil_18[[3]], digits = 0)
Create a 95\% confidence interval for the effect of fish oils
on heart attacks for patients who are well-represented by
those in the study.
Also interpret the interval in the context of the
study.\footnotemark
\end{nexercise}
\end{exercisewrap}
\footnotetext{
  Because the patients were randomized,
  the subjects are independent, both within and between
  the two groups.
  The success-failure condition is also met for both
  groups as all counts are at least~10.
  This satisfies the conditions necessary to model
  the difference in proportions using a normal distribution.

  Compute the sample proportions
  ($\hat{p}_{\text{fish oil}} = 0.0112$,
    $\hat{p}_{\text{placebo}} = 0.0155$),
  point estimate of the difference ($0.0112 - 0.0155 = -0.0043$),
  and standard error
  ($SE = \sqrt{\frac{0.0112 \times 0.9888}{12933} +
      \frac{0.0155 \times 0.9845}{12938}}
    = 0.00145$).
  Next, plug the values into the general formula for
  a confidence interval, where we'll use a 95\%
  confidence level with $z^{\star} = 1.96$:
  \begin{align*}
  -0.0043 \pm 1.96 \times 0.00145
      \quad \to \quad
      (-0.0071, -0.0015)
  \end{align*}
  We are 95\% confident that fish oils decreases
  heart attacks by
  0.15 to 0.71 percentage points
  (off of a baseline of about 1.55\%)
  over a 5-year period for subjects who are similar
  to those in the study.
  Because the interval is entirely below 0,
  the data provide strong evidence
  that fish oil supplements reduce heart attacks
  in patients like those in the study.}


\subsection{Hypothesis tests for $\mathbf{p_1 - p_2}$}

\index{data!mammography|(}
\index{data!breast cancer|(}

%We'll explore an experiment evaluating the benefits
%of mammograms using a hypothesis test.
A mammogram is an X-ray procedure used to check for breast cancer. Whether mammograms should be used is part of a controversial discussion, and it's the topic of our next example where we examine 2-proportion hypothesis test when $H_0$~is~$p_1 - p_2 = 0$ (or equivalently, $p_1 = p_2$).

A 30-year study was conducted with nearly 90,000 female participants. During a 5-year screening period, each woman was randomized to one of two groups: in the first group, women received regular mammograms to screen for breast cancer, and in the second group, women received regular non-mammogram breast cancer exams. No intervention was made during the following 25 years of the study, and we'll consider death resulting from breast cancer over the full 30-year period. Results from the study are summarized in Figure~\ref{mammogramStudySummaryTable}.

If mammograms are much more effective than non-mammogram breast cancer exams, then we would expect to see additional deaths from breast cancer in the control group. On~the other hand, if mammograms are not as effective as regular breast cancer exams, we~would expect to see an increase in breast cancer deaths in the mammogram group.

\begin{figure}[h]
\centering
\begin{tabular}{rrcc}
	& \multicolumn{3}{c}{Death from breast cancer?} \\
  \cline{2-4}
 & \ \hspace{3mm}\ & Yes & No \\
  \hline
Mammogram && 500 & 44,425 \\
Control && 505 & 44,405 \\
   \hline
\end{tabular}
\caption{Summary results for breast cancer study.}
\label{mammogramStudySummaryTable}
\end{figure}

\begin{exercisewrap}
\begin{nexercise}
Is this study an experiment or an observational study?\footnotemark
\end{nexercise}
\end{exercisewrap}
\footnotetext{This is an experiment. Patients were randomized
    to receive mammograms or a standard breast cancer exam.
    We will be able to make causal conclusions based on this study.}

\D{\newpage}

\begin{exercisewrap}
\begin{nexercise} \label{htFormammogramStudySummaryTable}
Set up hypotheses to test whether there was a difference
in breast cancer deaths in the mammogram and control groups.\footnotemark
\end{nexercise}
\end{exercisewrap}
\footnotetext{$H_0$: the breast cancer death rate for patients
    screened using mammograms is the same as the breast cancer
    death rate for patients in the control,
    $p_{mgm} - p_{ctrl} = 0$. \\
    $H_A$: the breast cancer death rate for patients screened
    using mammograms is different than the breast cancer death
    rate for patients in the control,
    $p_{mgm} - p_{ctrl} \neq 0$.}

In Example~\ref{condFormammogramStudySummaryTableNormalInference},
we will check the conditions for using a normal distribution to
analyze the results of the study.
The details are very similar to that of confidence intervals.
However, when the null hypothesis is that $p_1 - p_2 = 0$,
we use a special proportion called the
\term{pooled proportion} to check the success-failure condition:
\begin{align*}
\hat{p}_{\textit{pooled}}
    &= \frac
        {\text{\# of patients who died from breast cancer in the
            entire study}}
        {\text{\# of patients in the entire study}} \\
	&= \frac{500 + 505}{500 + \text{44,425} + 505 + \text{44,405}} \\
	&= 0.0112
\end{align*}
This proportion is an estimate of the breast cancer death rate
across the entire study, and it's our best estimate of the
proportions $p_{mgm}$ and $p_{ctrl}$
\emph{if the null hypothesis is true that $p_{mgm} = p_{ctrl}$}.
We~will also use this pooled proportion when computing
the standard error.

\begin{examplewrap}
\begin{nexample}{Can we use a normal model to analyze this study?}
  \label{condFormammogramStudySummaryTableNormalInference}%
  Because the patients are randomized, they can be treated
  as independent, both within and between groups.
  We also must check the success-failure condition for each group.
  Under the null hypothesis, the proportions $p_{mgm}$
  and $p_{ctrl}$ are equal, so we check the success-failure
  condition with our best estimate of these values under $H_0$,
  the \hiddenterm{pooled proportion} from the two samples,
  $\hat{p}_{\textit{pooled}} = 0.0112$:
  \begin{align*}
  \hat{p}_{\textit{pooled}} \times n_{mgm}
      &= 0.0112 \times \text{44,925} = 503
    & (1 - \hat{p}_{\textit{pooled}}) \times n_{mgm}
      &= 0.9888 \times \text{44,925} = \text{44,422} \\
  \hat{p}_{\textit{pooled}} \times n_{ctrl}
      &= 0.0112 \times \text{44,910} = 503
    & (1 - \hat{p}_{\textit{pooled}}) \times n_{ctrl}
      &= 0.9888 \times \text{44,910} = \text{44,407}
  \end{align*}
  With independence (within and between groups) and
  the success-failure condition satisfied since all values
  are at least 10, and we can safely use a normal distribution.
\end{nexample}
\end{examplewrap}

\begin{onebox}{Use the pooled proportion estimate when
    $\mathbf{H_0}$ is $\mathbf{p_1 - p_2 = 0}$}
  When the null hypothesis is that the proportions are equal,
  use the pooled proportion ($\hat{p}_{\textit{pooled}}$)
  to verify the
  success-failure condition and estimate the standard error:
  \begin{eqnarray*}
  \hat{p}_{\textit{pooled}}
    = \frac{\text{number of ``successes''}}
      {\text{number of cases}}
    = \frac{\hat{p}_1 n_1 + \hat{p}_2 n_2}{n_1 + n_2}
  \end{eqnarray*}
  Here $\hat{p}_1 n_1$ represents the number of successes in
  sample 1 since
  \begin{eqnarray*}
  \hat{p}_1
    = \frac{\text{number of successes in sample 1}}{n_1}
  \end{eqnarray*}
  Similarly, $\hat{p}_2 n_2$ represents the number of successes
  in sample 2.
\end{onebox}

In Example~\ref{condFormammogramStudySummaryTableNormalInference},
the pooled proportion was used to check the success-failure
condition.\footnote{For an example of a two-proportion
  hypothesis test that does not require the the
  success-failure condition to be met, see
  Section~\ref{caseStudyMalariaVaccine}.}
In the next example, we see the second place where the pooled
proportion comes into play: the standard error calculation.

\begin{examplewrap}
\begin{nexample}{Compute the point estimate of the difference
    in breast cancer death rates in the two groups,
    and use the pooled proportion
    $\hat{p}_{\textit{pooled}} = 0.0112$ to calculate
    the standard error.}
  The point estimate of the difference in breast cancer death
  rates is
  \begin{align*}
  \hat{p}_{mgm} - \hat{p}_{ctrl}
    &= \frac{500}{500 + 44,425} - \frac{505}{505 + 44,405} \\
    &= 0.01113 - 0.01125 \\
    &= -0.00012
  \end{align*}
  The breast cancer death rate in the mammogram group
  was 0.012\% less than in the control group.
  Next, the standard error is calculated
  \emph{using the pooled proportion},~$\hat{p}_{\textit{pooled}}$:
\begin{align*}
SE = \sqrt{
      \frac{\hat{p}_{\textit{pooled}}(1-\hat{p}_{\textit{pooled}})}
          {n_{mgm}}
      + \frac{\hat{p}_{\textit{pooled}}(1-\hat{p}_{\textit{pooled}})}
          {n_{ctrl}}
    }
	= 0.00070
\end{align*}
\end{nexample}
\end{examplewrap}

\begin{examplewrap}
\begin{nexample}{Using the point estimate $\hat{p}_{mgm} - \hat{p}_{ctrl} = -0.00012$ and standard error $SE = 0.00070$, calculate a p-value for the hypothesis test and write a conclusion.}
Just like in past tests, we first compute a test statistic and draw a picture:
\begin{align*}
Z = \frac{\text{point estimate} - \text{null value}}{SE}
	= \frac{-0.00012 - 0}{0.00070}
	= -0.17
\end{align*}
\begin{center}
\Figures{0.5}{mammograms}{mammogramPValue}
\end{center}
The lower tail area is 0.4325, which we double to get the p-value:~0.8650. Because this p-value is larger than 0.05, we do not reject the null hypothesis. That is, the difference in breast cancer death rates is reasonably explained by chance, and we do not observe benefits or harm from mammograms relative to a regular breast exam.
\end{nexample}
\end{examplewrap}

Can we conclude that mammograms have no benefits or harm?
Here are a few considerations to keep in mind when reviewing
the mammogram study as well as any other medical study:
\begin{itemize}
\setlength{\itemsep}{0mm}
\item If mammograms are helpful or harmful, the data suggest the effect isn't very large. So while we do not accept the null hypothesis, we also don't have sufficient evidence to conclude that mammograms reduce or increase breast cancer deaths.
\item Are mammograms more or less expensive than a non-mammogram breast exam? If~one option is much more expensive than the other and doesn't offer clear benefits, then we should lean towards the less expensive option.
\item The study's authors also found that mammograms led to overdiagnosis of breast cancer, which means some breast cancers were found (or thought to be found) but that these cancers would not cause symptoms during patients' lifetimes. That is, something else would kill the patient before breast cancer symptoms appeared. This means some patients may have been treated for breast cancer unnecessarily, and this treatment is another cost to consider. It is also important to recognize that overdiagnosis can cause unnecessary physical or emotional harm to patients.
\end{itemize}
These considerations highlight the complexity around medical care and treatment recommendations. Experts and medical boards who study medical treatments use considerations like those above to provide their best recommendation based on the current evidence.

\index{data!breast cancer|)}
\index{data!mammography|)}

%\begin{onebox}{Hypothesis testing when $\mathbf{H_0}$ is
%    $\mathbf{p_1 - p_2 = 0}$}
%  Once you've determined a hypothesis test for the difference
%  of two proportions is the correct procedure, there are four
%  steps to completing the test:
%  \begin{description}
%  \item[Prepare.]
%      Identify the parameter of interest,
%      list out hypotheses,
%      identify the significance level,
%      and compute summary statistics for each group.
%  \item[Check.]
%      Verify the conditions to ensure
%      $\hat{p}_1 - \hat{p}_2$ is nearly normal under $H_0$.
%      When the null hypothesis is that the difference is~0,
%      use a pooled proportion to check the success-failure
%      condition for each group.
%  \item[Calculate.]
%      If the conditions hold, compute the standard
%      error, again using the pooled proportion,
%      compute the Z-score, and identify the p-value.
%  \item[Conclude.]
%      Evaluate the hypothesis test by comparing the p-value
%      to $\alpha$, and provide a conclusion in the context
%      of the problem.
%  \end{description}
%\end{onebox}



\D{\newpage}

\subsection{More on 2-proportion hypothesis tests (special topic)}

When we conduct a 2-proportion hypothesis test,
usually $H_0$ is $p_1 - p_2 = 0$. However, there are rare
situations where we want to check for some difference in
$p_1$ and $p_2$ that is some value other than 0.
For example, maybe we care about checking a null hypothesis
where $p_1 - p_2 = 0.1$. %\footnote{We can
%  also encounter a similar situation with a difference of
%  two means, though no such example is given in
%  Chapter~\ref{inferenceForNumericalData} since the methods
%  remain exactly the same in the context of sample means.
%  On the other hand, the success-failure condition and the
%  calculation of the standard error vary slightly in different
%  proportion contexts.}
In contexts like these, we generally use $\hat{p}_1$ and
$\hat{p}_2$ to check the success-failure condition and
construct the standard error.

\begin{exercisewrap}
\begin{nexercise}
\label{carWheelBladeManufacturer}%
A quadcopter company is considering a new manufacturer
for rotor blades.
The new manufacturer would be more expensive but claims
their higher-quality blades are more reliable,
with 3\% more blades passing inspection than their
competitor.
Set up appropriate hypotheses for the test.\footnotemark
\end{nexercise}
\end{exercisewrap}
\footnotetext{$H_0$: The higher-quality blades will pass
  inspection 3\% more frequently than the standard-quality blades.
  $p_{highQ} - p_{standard} = 0.03$.
  $H_A$: The higher-quality blades will pass inspection
  some amount different than 3\% more often than the
  standard-quality blades.
  $p_{highQ} - p_{standard} \neq 0.03$.}

\setlength{\captionwidth}{85mm}

\begin{figure}[h]
\centering
\Figures{0.6}{quadcopter}{quadcopter_david_j}
\caption{A Phantom quadcopter.\vspace{-1mm} \\
   -----------------------------\vspace{-2mm}\\
   {\footnotesize Photo by David J
   (\oiRedirect{textbook-quadcopter_david_j}
       {http://flic.kr/p/oiWLNu}).
   \oiRedirect{textbook-CC_BY_2}{CC-BY 2.0 license.}
   This photo has been cropped and a border has been added.}}
\label{quadcopter_david_j}
\end{figure}

\setlength{\captionwidth}{\mycaptionwidth}

%\Add{In Guided Practice~\ref{qualityCtrlEngHypothesisEval}, the null difference is 0.03. However, in the vast majority of applications for differences in means or proportions, the null difference is~0. While the details for a difference of means does not change if the null difference is zero or non-zero, that is not the case for a difference in proportions. As we'll see in Section~\ref{}, a hypothesis test for a difference in proportions where the null value is 0 requires additional~care.}

\begin{examplewrap}
\begin{nexample}{The quality control engineer from
    Guided Practice~\ref{carWheelBladeManufacturer}
    collects a sample of blades, examining 1000 blades
    from each company and finds that 899 blades pass
    inspection from the current supplier and 958 pass
    inspection from the prospective supplier.
    Using these data, evaluate the hypothesis setup of
    Guided Practice~\ref{carWheelBladeManufacturer}
    with a significance level of 5\%.}
  \label{qualityCtrlEngHypothesisEval}%
  First, we check the conditions.
  The sample is not necessarily random, so to proceed
  we must assume the blades are all independent;
  for this sample we will suppose this assumption
  is reasonable, but the engineer would be more knowledgeable
  as to whether this assumption is appropriate.
  The success-failure condition also holds for each sample.
  Thus, the difference in sample proportions,
  $0.958 - 0.899 = 0.059$, can be said to come from a nearly
  normal distribution.

  The standard error is computed using the two sample
  proportions since we do not use a pooled proportion
  for this context:
  \begin{align*}
  SE
    = \sqrt{\frac{0.958(1-0.958)}{1000} +
        \frac{0.899(1-0.899)}{1000}}
    = 0.0114
  \end{align*}
  In this hypothesis test, because the null is that
  $p_1 - p_2 = 0.03$, the sample proportions were used
  for the standard error calculation rather than a pooled
  proportion.

  Next, we compute the test statistic and use it to find the
  p-value, which is depicted in
  Figure~\ref{bladesTwoSampleHTPValueQC}.
  \begin{align*}
  Z = \frac{\text{point estimate} - \text{null value}}{SE}
    = \frac{0.059 - 0.03}{0.0114} = 2.54
  \end{align*}
  Using a standard normal distribution for this test statistic,
  we identify the right tail area as 0.006,
  and we double it to get the p-value: 0.012.
  We reject the null hypothesis because 0.012 is less than 0.05.
  Since we observed a larger-than-3\% increase in blades
  that pass inspection, we have statistically significant
  evidence that the higher-quality blades pass inspection
  \emph{more than} 3\% as often as the currently used blades,
  exceeding the company's claims.
\end{nexample}
\end{examplewrap}

\begin{figure}[h]
  \centering
  \Figure{0.5}{bladesTwoSampleHTPValueQC}
  \caption{Distribution of the test statistic if the null
      hypothesis was true.
      The p-value is represented by the shaded areas.}
  \label{bladesTwoSampleHTPValueQC}
\end{figure}


\D{\newpage}

\subsection{Examining the standard error formula (special topic)}

The formula for the standard error of the difference
in two proportions can be deconstructed into the formulas
for the standard errors of the individual sample proportions.
Recall that the standard error of a single proportion,
$\hat{p}_1$, is
\begin{align*}
SE_{\hat{p}_1} = \sqrt{\frac{{p}_1 (1 - {p}_1)}{n_1}}
\end{align*}
The standard error of the difference of two sample proportions
can be constructed from the standard errors of the separate
sample proportions:
\begin{align*}
SE_{\hat{p}_{1} - \hat{p}_{2}}
	= \sqrt{SE_{\hat{p}_1}^2 + SE_{\hat{p}_2}^2}
	= \sqrt{\frac{{p}_1 (1 - {p}_1)}{n_1}
	    + \frac{{p}_2 (1 - {p}_2)}{n_2}}
\end{align*}
This special relationship follows from probability theory.

\begin{exercisewrap}
\begin{nexercise}
\label{derivingSEForDiffOfTwoMeansExercise}%
Prerequisite: Section~\ref{randomVariablesSection}.
We can rewrite the equation above in a different way:
\begin{align*}
SE_{\hat{p}_{1} - \hat{p}_{2}}^2
  = SE_{\hat{p}_1}^2 + SE_{\hat{p}_2}^2
\end{align*}
Explain where this formula comes from using the ideas of probability theory.\footnotemark{}
\end{nexercise}
\end{exercisewrap}
\footnotetext{The standard error squared represents
  the variance of the estimate.
  If $X$ and $Y$ are two random variables with variances
  $\sigma_x^2$ and $\sigma_y^2$,
  then the variance of $X - Y$ is $\sigma_x^2 + \sigma_y^2$.
  Likewise, the variance corresponding to $\hat{p}_1 - \hat{p}_2$ is $\sigma_{\hat{p}_1}^2 + \sigma_{\hat{p}_2}^2$. Because $\sigma_{\hat{p}_1}^2$ and $\sigma_{\hat{p}_2}^2$ are just another way of writing $SE_{\hat{p}_1}^2$ and  $SE_{\hat{p}_2}^2$, the variance associated with $\hat{p}_1 - \hat{p}_2$ may be written as $SE_{\hat{p}_1}^2 + SE_{\hat{p}_2}^2$.}



%%__________________
%\section{Determining a sample size for an experiment}
%\label{SampleSizeFor2Proportions}
%
%So far we've been focused on controlling the Type~1 Error rate for hypothesis tests. However, when planning an experiment, we often are interested in determining if there is an effect.\footnote{Similar planning is also appropriate for a} There are often two competing considerations:
%\begin{itemize}
%\setlength{\itemsep}{0mm}
%\item We want to collect enough data that we can detect important effects.
%\item In many contexts, collecting data is expensive, so we don't want to collect more than what we need to detect effects we care about.
%\end{itemize}
%The first point is relatively simple: the more data we collect, the more precise our estimates will be, and so we'll be able to detect smaller effects. The second point is more subtle, since we need to determine the size of effects that we care about.
%
%\begin{examplewrap}
%\begin{nexample}{Alzheimer's disease is a neurological disease. It affects patients mildly at the beginning and eventually leads to dementia. If an Alzheimer's patient lives long enough, the disease will begin affecting bodily functions and ultimately lead to death. It's an extremely serious condition that millions of people, has no cure, and is very expensive to research, partially due to its slow progression. A group of researchers is }
%\end{nexample}
%\end{examplewrap}
%
%
%, even large ones, are difficult to detect with small samples, so we should want to collect a larger sample to detect such effects. If we take a very large sample, we might find a statistically significant difference but the magnitude might be so small that it is of no practical value. In this section we describe techniques for selecting an appropriate sample size based on these considerations.



%__________________
\section{Testing for goodness of fit using chi-square}
\label{oneWayChiSquare}

In this section, we develop a method for assessing a null model when the data are binned.
This technique is commonly used in two circumstances:
\begin{itemize}
\setlength{\itemsep}{0mm}
\item Given a sample of cases that can be classified into several groups, determine if the sample is representative of the general population.
\item Evaluate whether data resemble a particular distribution, such as a normal distribution or a geometric distribution.
\end{itemize}
Each of these scenarios can be addressed using the same statistical test: a chi-square test.

\index{data!racial make-up of jury|(}

In the first case, we consider data from a random sample of 275 jurors in a small county. Jurors identified their racial group, as shown in Figure~\ref{juryRepresentationAndCityRepresentationForRace}, and we would like to determine if these jurors are racially representative of the population.  If the jury is representative of the population, then the proportions in the sample should roughly reflect the population of eligible jurors, i.e. registered voters.

\begin{figure}[h]
\centering
\begin{tabular}{ll ccc c ll}
\hline
Race	 & \hspace{2mm} & White & Black & Hispanic & Other & \hspace{2mm} & Total \\
\hline
Representation in juries &	& 205 & 26 & 25 & 19 & & 275 \\
Registered voters	 & 		& 0.72 & 0.07 & 0.12 & 0.09 & & 1.00 \\
\hline
\end{tabular}
\caption{Representation by race in a city's juries and population.}
\label{juryRepresentationAndCityRepresentationForRace}
\end{figure}

While the proportions in the juries do not precisely represent the population proportions, it is unclear whether these data provide convincing evidence that the sample is not representative. If the jurors really were randomly sampled from the registered voters, we might expect small differences due to chance. However, unusually large differences may provide convincing evidence that the juries were not representative.

A second application, assessing the fit of a distribution, is presented at the end of this section. Daily stock returns from the S\&P500 for 25 years are used to assess whether stock activity each day is independent of the stock's behavior on previous days.

In these problems, we would like to examine all bins simultaneously, not simply compare one or two bins at a time, which will require us to develop a new test statistic.


\subsection{Creating a test statistic for one-way tables}

\begin{examplewrap}
\begin{nexample}{Of the people in the city, 275 served on a jury. If the individuals are randomly selected to serve on a jury, about how many of the 275 people would we expect to be white? How many would we expect to be black?}
About 72\% of the population is white, so we would expect about 72\% of the jurors to be white: $0.72\times 275 = 198$.

Similarly, we would expect about 7\% of the jurors to be black, which would correspond to about $0.07\times 275 = 19.25$ black jurors.
\end{nexample}
\end{examplewrap}

\begin{exercisewrap}
\begin{nexercise}
Twelve percent of the population is Hispanic and 9\% represent other races. How many of the 275 jurors would we expect to be Hispanic or from another race? Answers can be found in Figure~\ref{expectedJuryRepresentationIfNoBias}.
\end{nexercise}
\end{exercisewrap}

\begin{figure}[h]
\centering
\begin{tabular}{ll ccc c ll}
\hline
Race	 & \hspace{2mm} & White & Black & Hispanic & Other & \hspace{2mm} & Total \\
\hline
Observed data			&	& 205 & 26	& 25 & 19	&	& 275 \\
Expected counts	 &	& 198 & 19.25 & 33 & 24.75 & & 275 \\
\hline
\end{tabular}
\caption{Actual and expected make-up of the jurors.}
\label{expectedJuryRepresentationIfNoBias}
\end{figure}

The sample proportion represented from each race among the 275 jurors was not a precise match for any ethnic group. While some sampling variation is expected, we would expect the sample proportions to be fairly similar to the population proportions if there is no bias on juries. We need to test whether the differences are strong enough to provide convincing evidence that the jurors are not a random sample. These ideas can be organized into hypotheses:
\begin{itemize}
\setlength{\itemsep}{0mm}
\item[$H_0$:] The jurors are a random sample, i.e. there is no racial bias in who serves on a jury, and the observed counts reflect natural sampling fluctuation.
\item[$H_A$:] The jurors are not randomly sampled, i.e. there is racial bias in juror selection.
\end{itemize}
To evaluate these hypotheses, we quantify how different the observed counts are from the expected counts. Strong evidence for the alternative hypothesis would come in the form of unusually large deviations in the groups from what would be expected based on sampling variation alone.


\subsection{The chi-square test statistic}
\label{chiSquareTestStatistic}

In previous hypothesis tests, we constructed a test statistic of the following form:
\begin{align*}
\frac{\text{point estimate} - \text{null value}}
    {\text{SE of point estimate}}
\end{align*}
This construction was based on (1) identifying the difference between a point estimate and an expected value if the null hypothesis was true, and (2) standardizing that difference using the standard error of the point estimate. These two ideas will help in the construction of an appropriate test statistic for count data.

Our strategy will be to first compute the difference between the observed counts and the counts we would expect if the null hypothesis was true, then we will standardize the difference:
\begin{align*}
Z_{1} = \frac{\text{observed white count} - \text{null white count}}
				{\text{SE of observed white count}}
\end{align*}
The standard error for the point estimate of the count in binned data is the square root of the count under the null.\footnote{Using some of the rules learned in earlier chapters, we might think that the standard error would be $np(1-p)$, where $n$ is the sample size and $p$ is the proportion in the population. This would be correct if we were looking only at one count. However, we are computing many standardized differences and adding them together. It can be shown -- though not here -- that the square root of the count is a better way to standardize the count differences.} Therefore:
\begin{align*}
Z_1 = \frac{205 - 198}{\sqrt{198}} = 0.50
\end{align*}
The fraction is very similar to previous test statistics: first compute a difference, then standardize it. These computations should also be completed for the black, Hispanic, and other groups:
\begin{align*}
&Black && Hispanic	&&Other \\
& Z_2 = \frac{26-19.25}{\sqrt{19.25}}=1.54\ \ \ \ 
	&& Z_3 = \frac{25-33}{\sqrt{33}}=-1.39\ \ \ \ 
	&& Z_4 = \frac{19-24.75}{\sqrt{24.75}}=-1.16 \\
\end{align*}
We would like to use a single test statistic to determine if these four standardized differences are irregularly far from zero. That is, $Z_1$, $Z_2$, $Z_3$, and $Z_4$ must be combined somehow to help determine if they -- as a group -- tend to be unusually far from zero. A first thought might be to take the absolute value of these four standardized differences and add them~up:
\begin{align*}
|Z_1| + |Z_2| + |Z_3| + |Z_4| = 4.58
\end{align*}
Indeed, this does give one number summarizing how far the actual counts are from what was expected. However, it is more common to add the squared values:
\begin{align*}
Z_1^2 + Z_2^2 + Z_3^2 + Z_4^2 = 5.89
\end{align*}
Squaring each standardized difference before adding them together does two things:
\begin{itemize}
\setlength{\itemsep}{0mm}
\item Any standardized difference that is squared will now be positive.
\item Differences that already look unusual -- e.g. a standardized difference of 2.5 -- will become much larger after being squared.
\end{itemize}
The test statistic $X^2$,\index{chi-square statistic} which is the sum of the $Z^2$ values, is generally used for these reasons. We can also write an equation for $X^2$ using the observed counts and null counts:
\index{data!racial make-up of jury|)}
\begin{align*}
X^2 &=
	\frac
	{\text{\footnotesize$(\text{observed count}_1 - \text{null count}_1)^2$}}
	{\text{\footnotesize$\text{null count}_1$}}
	+ \dots + \frac
	{\text{\footnotesize$(\text{observed count}_4 - \text{null count}_4)^2$}}
	{\text{\footnotesize$\text{null count}_4$}}
\end{align*}
The final number $X^2$ summarizes how strongly the observed counts tend to deviate from the null counts. In Section~\ref{pValueForAChiSquareTest}, we will see that if the null hypothesis is true, then $X^2$ follows a new distribution called a \emph{chi-square distribution}. Using this distribution, we will be able to obtain a p-value to evaluate the hypotheses.


\subsection{The chi-square distribution and finding areas}

The \term{chi-square distribution} is sometimes used to
characterize data sets and statistics that are always positive
and typically right skewed. Recall a normal distribution had
two parameters -- mean and standard deviation -- that could be
used to describe its exact characteristics.
The chi-square distribution has just one parameter called
\termsub{degrees of freedom (df)}{degrees of freedom (df)!chi-square},
which influences the shape, center, and spread of the distribution.

\begin{exercisewrap}
\begin{nexercise}\label{exerChiSquareDistributionDescriptionWithMoreDOF}
Figure~\ref{chiSquareDistributionWithInceasingDF} shows three chi-square distributions. (a) How does the center of the distribution change when the degrees of freedom is larger? (b) What about the variability (spread)? (c) How does the shape change?\footnotemark
\end{nexercise}
\end{exercisewrap}
\footnotetext{(a)~The center becomes larger. If took a careful look, we could see that the mean of each distribution is equal to the distribution's degrees of freedom. (b)~The variability increases as the degrees of freedom increases. (c)~The distribution is very strongly skewed for $df=2$, and then the distributions become more symmetric for the larger degrees of freedom $df=4$ and $df=9$. We would see this trend continue if we examined distributions with even more larger degrees of freedom.}

\begin{figure}[h]
\centering
\includegraphics[width=0.9\textwidth]{ch_inference_for_props/figures/chiSquareDistributionWithInceasingDF/chiSquareDistributionWithInceasingDF}
\caption{Three chi-square distributions with varying degrees of freedom.}
\label{chiSquareDistributionWithInceasingDF}
\end{figure}

Figure~\ref{chiSquareDistributionWithInceasingDF} and Guided Practice~\ref{exerChiSquareDistributionDescriptionWithMoreDOF} demonstrate three general properties of chi-square distributions as the degrees of freedom increases: the distribution becomes more symmetric, the center moves to the right, and the variability inflates.

Our principal interest in the chi-square distribution
is the calculation of p-values, which (as we have seen before)
is related to finding the relevant area in the tail of
a distribution.
The most common ways to do this are using computer software,
using a graphing calculator, or using a table.
For folks wanting to use the table option,
we provide an outline of how to read the chi-square table in
Appendix~\ref{chiSquareProbabilityTable},
which is also where you may find the table.
\Comment{If giving some \R{} in the text, then put \R{} code
  in the examples / exercises below.}
For the examples below, use your preferred approach
to confirm you get the same answers.

\begin{examplewrap}
\begin{nexample}{Figure~\ref{chiSquareAreaAbove6Point25WithDF3}
    shows a chi-square distribution with 3 degrees of freedom
    and an upper shaded tail starting at 6.25.
    Find the shaded area.}
  Using statistical software or a graphing calculator,
  we can find that the upper tail area for a chi-square
  distribution with 3 degrees of freedom ($df$)
  and a cutoff of 6.25 is 0.1001.
  That is, the shaded upper tail of
  Figure~\ref{chiSquareAreaAbove6Point25WithDF3}
  has area 0.1.
\end{nexample}
\end{examplewrap}

\begin{figure}
\centering
\subfigure[]{
\includegraphics[width=0.475\textwidth]{ch_inference_for_props/figures/arrayOfFigureAreasForChiSquareDistribution/chiSquareAreaAbove6Point25WithDF3/chiSquareAreaAbove6Point25WithDF3}
\label{chiSquareAreaAbove6Point25WithDF3}
}
\subfigure[]{
\includegraphics[width=0.475\textwidth]{ch_inference_for_props/figures/arrayOfFigureAreasForChiSquareDistribution/chiSquareAreaAbove4Point3WithDF2/chiSquareAreaAbove4Point3WithDF2}
\label{chiSquareAreaAbove4Point3WithDF2}
}
\subfigure[]{
\includegraphics[width=0.475\textwidth]{ch_inference_for_props/figures/arrayOfFigureAreasForChiSquareDistribution/chiSquareAreaAbove5Point1WithDF5/chiSquareAreaAbove5Point1WithDF5}
\label{chiSquareAreaAbove5Point1WithDF5}
}
\subfigure[]{
\includegraphics[width=0.475\textwidth]{ch_inference_for_props/figures/arrayOfFigureAreasForChiSquareDistribution/chiSquareAreaAbove11Point7WithDF7/chiSquareAreaAbove11Point7WithDF7}
\label{chiSquareAreaAbove11Point7WithDF7}
}
\subfigure[]{
\includegraphics[width=0.475\textwidth]{ch_inference_for_props/figures/arrayOfFigureAreasForChiSquareDistribution/chiSquareAreaAbove10WithDF4/chiSquareAreaAbove10WithDF4}
\label{chiSquareAreaAbove10WithDF4}
}
\subfigure[]{
\includegraphics[width=0.475\textwidth]{ch_inference_for_props/figures/arrayOfFigureAreasForChiSquareDistribution/chiSquareAreaAbove9Point21WithDF3/chiSquareAreaAbove9Point21WithDF3}
\label{chiSquareAreaAbove9Point21WithDF3}
}
\caption{
\textbf{\subref{chiSquareAreaAbove6Point25WithDF3}}~Chi-square distribution with 3~degrees of freedom, area above 6.25 shaded.
\textbf{\subref{chiSquareAreaAbove4Point3WithDF2}}~2~degrees of freedom, area above 4.3 shaded.
\textbf{\subref{chiSquareAreaAbove5Point1WithDF5}}~5~degrees of freedom, area above 5.1 shaded.
\textbf{\subref{chiSquareAreaAbove11Point7WithDF7}}~7~degrees of freedom, area above 11.7 shaded.
\textbf{\subref{chiSquareAreaAbove10WithDF4}}~4~degrees of freedom, area above 10 shaded.
\textbf{\subref{chiSquareAreaAbove9Point21WithDF3}}~3~degrees of freedom, area above 9.21 shaded.
}
\label{arrayOfFigureAreasForChiSquareDistribution}
\end{figure}

\begin{examplewrap}
\begin{nexample}{Figure~\ref{chiSquareAreaAbove4Point3WithDF2}
    shows the upper tail of a chi-square distribution with
    2~degrees of freedom.
    The bound for this upper tail is at 4.3.
    Find the tail area.}
  Using software, we can find that the tail area shaded in
  Figure~\ref{chiSquareAreaAbove4Point3WithDF2}
  to be 0.1165.
  If using a table, we would only be able to find
  a range of values for the tail area:
  between 0.1 and 0.2.
\end{nexample}
\end{examplewrap}

\begin{examplewrap}
\begin{nexample}{Figure~\ref{chiSquareAreaAbove5Point1WithDF5}
    shows an upper tail for a chi-square distribution with
    5~degrees of freedom and a cutoff of 5.1.
    Find the tail area.}
  Using software, we would obtain a tail area of 0.4038.
  If using the table in Appendix~\ref{chiSquareProbabilityTable},
  we would have identified that the tail area is larger than 0.3
  but not be able to give the precise value.
\end{nexample}
\end{examplewrap}

\begin{exercisewrap}
\begin{nexercise}
Figure~\ref{chiSquareAreaAbove11Point7WithDF7} shows a cutoff
of 11.7 on a chi-square distribution with 7 degrees of freedom.
Find the area of the upper tail.\footnotemark{}
\end{nexercise}
\end{exercisewrap}
\footnotetext{
  The area is 0.1109.
  If using a table, we would identify that it falls
  between 0.1 and 0.2.}

\begin{exercisewrap}
\begin{nexercise}
Figure~\ref{chiSquareAreaAbove10WithDF4} shows a cutoff
of 10 on a chi-square distribution with 4 degrees of freedom.
Find the area of the upper tail.\footnotemark{}
\end{nexercise}
\end{exercisewrap}
\footnotetext{Precise value: 0.0404.
  If using the table: between 0.02 and 0.05.}

\begin{exercisewrap}
\begin{nexercise}
Figure~\ref{chiSquareAreaAbove9Point21WithDF3} shows a cutoff
of 9.21 with a chi-square distribution with 3 df.
Find the area of the upper tail.\footnotemark{}
\end{nexercise}
\end{exercisewrap}
\footnotetext{Precise value: 0.0266.
  If using the table: between 0.02 and 0.05.}


\D{\newpage}

\subsection{Finding a p-value for a chi-square distribution}
\label{pValueForAChiSquareTest}

\index{data!racial make-up of jury|(}
In Section~\ref{chiSquareTestStatistic}, we identified a new test statistic ($X^2$) within the context of assessing whether there was evidence of racial bias in how jurors were sampled. The null hypothesis represented the claim that jurors were randomly sampled and there was no racial bias. The alternative hypothesis was that there was racial bias in how the jurors were sampled.

We determined that a large $X^2$ value would suggest strong evidence favoring the alternative hypothesis: that there was racial bias. However, we could not quantify what the chance was of observing such a large test statistic ($X^2=5.89$) if the null hypothesis actually was true. This is where the chi-square distribution becomes useful. If the null hypothesis was true and there was no racial bias, then $X^2$ would follow a chi-square distribution, with three degrees of freedom in this case. Under certain conditions, the statistic $X^2$ follows a chi-square distribution with $k - 1$ degrees of freedom, where $k$ is the number of bins.

\begin{examplewrap}
\begin{nexample}{How many categories were there in the juror example? How many degrees of freedom should be associated with the chi-square distribution used for $X^2$?}
In the jurors example, there were $k=4$ categories: white, black, Hispanic, and other. According to the rule above, the test statistic $X^2$ should then follow a chi-square distribution with $k-1 = 3$ degrees of freedom if $H_0$ is true.
\end{nexample}
\end{examplewrap}

Just like we checked sample size conditions to use a normal
distribution in earlier sections, we must also check a sample
size condition to safely apply the chi-square distribution
for~$X^2$.
Each expected count must be at least 5. In the juror example,
the expected counts were 198, 19.25, 33, and 24.75, all easily
above~5, so we can apply the chi-square model to the test
statistic, $X^2=5.89$.

\begin{examplewrap}
\begin{nexample}{If the null hypothesis is true,
    the test statistic $X^2=5.89$ would be closely
    associated with a chi-square distribution with
    three degrees of freedom.
    Using this distribution and test statistic,
    identify the p-value.}
  The chi-square distribution and p-value are shown in
  Figure~\ref{jurorHTPValueShown}.
  Because larger chi-square values correspond to stronger
  evidence against the null hypothesis, we shade the upper
  tail to represent the p-value.
  Using statistical software (or the table in
  Appendix~\ref{chiSquareProbabilityTable}),
  we can determine that the area is 0.1171.
  Generally we do not reject the null hypothesis
  with such a large p-value.
  In other words, the data do not provide convincing evidence
  of racial bias in the juror selection.
\end{nexample}
\end{examplewrap}

\begin{figure}[h]
\centering
\includegraphics[width=0.55\textwidth]{ch_inference_for_props/figures/jurorHTPValueShown/jurorHTPValueShown}
\caption{The p-value for the juror hypothesis test is shaded in the chi-square distribution with $df=3$.}
\label{jurorHTPValueShown}
\end{figure}

\index{data!racial make-up of jury|)}

\begin{onebox}{Chi-square test for one-way table}
  Suppose we are to evaluate whether there is convincing
  evidence that a set of observed counts $O_1$, $O_2$, ...,
  $O_k$ in $k$ categories are unusually different from what
  might be expected under a null hypothesis.
  Call the \emph{expected counts} that are based on the null
  hypothesis $E_1$, $E_2$, ..., $E_k$.
  If each expected count is at least 5 and the null hypothesis
  is true, then the test statistic below follows a chi-square
  distribution with $k-1$ degrees of freedom:
  \begin{align*}
  X^2
    = \frac{(O_1 - E_1)^2}{E_1} +
        \frac{(O_2 - E_2)^2}{E_2} +
        \cdots +
        \frac{(O_k - E_k)^2}{E_k}
  \end{align*}
  The p-value for this test statistic is found by looking
  at the upper tail of this chi-square distribution.
  We consider the upper tail because larger values of $X^2$
  would provide greater evidence against the null hypothesis.
\end{onebox}

\begin{onebox}{Conditions for the chi-square test}
  There are two conditions that must be checked before
  performing a chi-square test:\vspace{-1mm}
  \begin{description}
  \setlength{\itemsep}{0mm}
  \item[Independence.] Each case that contributes a count to
      the table must be independent of all the other cases in
      the table.
  \item[Sample size / distribution.] Each particular scenario
      (i.e. cell count) must have at least 5~expected cases.
  \end{description}
  Failing to check conditions may affect the test's error rates.
\end{onebox}

%\begin{onebox}{Chi-square test for one-way table}
%  Suppose we are to evaluate whether there is convincing
%  evidence that a set of observed counts $O_1$, $O_2$, ...,
%  $O_k$ in $k$ categories are unusually different from what
%  might be expected under a null hypothesis.
%  \begin{description}
%  \item[Prepare.]
%      List out hypotheses and identify the significance level.
%  \item[Check.]
%      Verify the conditions are met,
%      which will include finding the expected value
%      for each of the $k$ cells based on the null hypothesis,
%      which we'll label as $E_1$, $E_2$, ..., $E_k$.
%  \item[Calculate.]
%      Compute the degrees of freedom $df = k - 1$ and
%      the test statistic using the expected values
%      against the observed values $O_1, ..., O_k$:
%      \begin{align*}
%      X^2
%        = \frac{(O_1 - E_1)^2}{E_1} +
%            \frac{(O_2 - E_2)^2}{E_2} +
%            \cdots +
%            \frac{(O_k - E_k)^2}{E_k}
%      \end{align*}
%      Identify the p-value as the upper tail in the chi-square
%      distribution using the test statistic as a cutoff.
%  \item[Conclude.]
%      Evaluate the hypothesis test by comparing the p-value
%      to $\alpha$, and provide a conclusion in the context
%      of the problem.
%  \end{description}
%\end{onebox}

When examining a table with just two bins,
pick a single bin and use the one-proportion methods
introduced in Section~\ref{singleProportion}.


\D{\newpage}

\subsection{Evaluating goodness of fit for a distribution}

Section~\ref{geomDist} would be useful background reading
for this example, but it is not a prerequisite.

\index{data!S\&P500 stock data|(}

\newcommand{\spyears}{10}
\newcommand{\spdays}{1362}
\newcommand{\spdaysA}{717}
\newcommand{\spdaysB}{369}
\newcommand{\spdaysC}{155}
\newcommand{\spdaysD}{69}
\newcommand{\spdaysE}{28}
\newcommand{\spdaysF}{14}
\newcommand{\spdaysG}{10}
\newcommand{\spdaysEA}{743}
\newcommand{\spdaysEB}{338}
\newcommand{\spdaysEC}{154}
\newcommand{\spdaysED}{70}
\newcommand{\spdaysEE}{32}
\newcommand{\spdaysEF}{14}
\newcommand{\spdaysEG}{12}
\newcommand{\spdaysEProp}{0.1128}
\newcommand{\spdaysEPerc}{11.28\%}
\newcommand{\spUpProp}{0.545}
\newcommand{\spUpPerc}{54.5\%}
\newcommand{\spDownProp}{0.455}
\newcommand{\spDownPerc}{45.5\%}
\newcommand{\spdaysXSq}{4.61}
\newcommand{\spdaysN}{7}
\newcommand{\spdaysDF}{6}
\newcommand{\spdaysPvalue}{0.5951}

We can apply the chi-square testing framework to the
second problem in this section:
evaluating whether a certain statistical model fits
a data set.
Daily stock returns from the S\&P500 for \spyears{} can be used to assess whether stock activity each day is independent of the stock's behavior on previous days. This sounds like a very complex question, and it is, but a chi-square test can be used to study the problem. We will label each day as \resp{Up} or \resp{Down} (\resp{D}) depending on whether the market was up or down that day. For example, consider the following changes in price, their new labels of up and down, and then the number of days that must be observed before each \resp{Up} day:
\begin{center}\footnotesize
\begin{tabular}{lc ccc ccc ccc cc}
Change in price		&\hspace{-1mm}	& \footnotesize2.52 &
	\footnotesize-1.46 & \footnotesize 0.51 &
	\footnotesize-4.07 & \footnotesize3.36 &
	\footnotesize1.10 &
	\footnotesize-5.46 & \footnotesize-1.03 & \footnotesize-2.99 & \footnotesize1.71 \\
Outcome	 & \hspace{-1mm} &
	Up &
	D & Up &
	D & Up &
	Up &
	D & D & D & Up \\
\footnotesize Days to Up & \hspace{-1mm} & 1 & - & 2 & - & 2 & 1 & - & - & - & 4 \\
\end{tabular}
\end{center}
If the days really are independent, then the number of days until a positive trading day should follow a geometric distribution. The geometric distribution describes the probability of waiting for the $k^{th}$ trial to observe the first success. Here each up day (Up) represents a success, and down (D) days represent failures. In the data above, it took only one day until the market was up, so the first wait time was 1 day. It took two more days before we observed our next \resp{Up} trading day, and two more for the third \resp{Up} day. We would like to determine if these counts (1, 2, 2, 1, 4, and so on) follow the geometric distribution. Figure~\ref{sAndP500TimeToPosTrade} shows the number of waiting days for a positive trading day during \spyears{} years for the S\&P500.

\begin{figure}[h]
\centering
\begin{tabular}{ll ccc ccc c ll}
  \hline
  Days	 & \hspace{2mm} & 1 & 2 & 3 & 4 & 5 & 6 & 7+ &
      \hspace{2mm} & Total \\
  Observed &		& \spdaysA{} & \spdaysB{} & \spdaysC{} &
      \spdaysD{} & \spdaysE{} & \spdaysF{} & \spdaysG{} & &
      \spdays{} \\
  \hline
\end{tabular}
\caption{Observed distribution of the waiting time until
    a positive trading day for the S\&P500.}
\label{sAndP500TimeToPosTrade}
\end{figure}

We consider how many days one must wait until observing an \resp{Up} day on the S\&P500 stock index. If the stock activity was independent from one day to the next and the probability of a positive trading day was constant, then we would expect this waiting time to follow a \emph{geometric distribution}. We can organize this into a hypothesis framework:
\begin{itemize}
\item[$H_0$:] The stock market being up or down on a given day is independent from all other days. We will consider the number of days that pass until an \resp{Up} day is observed. Under this hypothesis, the number of days until an \resp{Up} day should follow a geometric distribution.
\item[$H_A$:] The stock market being up or down on a given day is not independent from all other days. Since we know the number of days until an \resp{Up} day would follow a geometric distribution under the null, we look for deviations from the geometric distribution, which would support the alternative hypothesis.
\end{itemize}
There are important implications in our result for stock traders: if information from past trading days is useful in telling what will happen today, that information may provide an advantage over other traders.

We consider data for the S\&P500 and summarize the waiting times in Figure~\ref{sAndP500TimeToPosTrade2} and Figure~\ref{geomFitEvaluationForSP500}. The S\&P500 was positive on \spUpPerc{} of those days.

\begin{figure}
\centering
\begin{tabular}{ll ccc ccc c ll}
  \hline
  Days & \hspace{1mm} &
      1 & 2 & 3 & 4 & 5 & 6 & 7+ & \hspace{1mm} & Total \\
  \hline
  Observed & & \spdaysA{} & \spdaysB{} & \spdaysC{} &
      \spdaysD{} & \spdaysE{} & \spdaysF{} & \spdaysG{} & &
      \spdays{} \\
  Geometric Model & & \spdaysEA{} & \spdaysEB{} & \spdaysEC{} &
      \spdaysED{} & \spdaysEE{} & \spdaysEF{} & \spdaysEG{} & &
      \spdays{} \\
  \hline
\end{tabular}
\caption{Distribution of the waiting time until a positive
    trading day.
    The expected counts based on the geometric model are
    shown in the last row.
    To find each expected count, we identify the probability
    of waiting $D$ days based on the geometric model
    ($P(D) = (1-\spUpProp{})^{D-1}(\spUpProp{})$)
    and multiply by the total number of streaks, \spdays{}.
    For example, waiting for three days occurs under the
    geometric model about
    $\spDownProp{}^2\times \spUpProp{} = \spdaysEPerc{}$
    of the time, which corresponds to
    $\spdaysEProp{} \times \spdays{} = \spdaysEC$ streaks.}
\label{sAndP500TimeToPosTrade2}
\end{figure}

\begin{figure}
  \centering
  \Figure{0.85}{geomFitEvaluationForSP500}
  \caption{Side-by-side bar plot of the observed
      and expected counts for each waiting time.}
  \label{geomFitEvaluationForSP500}
\end{figure}

Because applying the chi-square framework requires expected counts
to be at least~5, we have \emph{binned} together all the cases
where the waiting time was at least \spdaysN{} days to ensure each
expected count is well above this minimum.
The actual data, shown in the \emph{Observed} row in
Figure~\ref{sAndP500TimeToPosTrade2}, can be compared to the
expected counts from the \emph{Geometric Model} row.
The method for computing expected counts is discussed in
Figure~\ref{sAndP500TimeToPosTrade2}.
In general, the expected counts are determined by
(1)~identifying the null proportion associated with each bin,
then (2)~multiplying each null proportion by the total count
to obtain the expected counts.
That is, this strategy identifies what proportion of the total
count we would expect to be in each bin.

\begin{examplewrap}
\begin{nexample}{Do you notice any unusually large deviations
    in the graph?
    Can you tell if these deviations are due to chance just
    by looking?}
  It is not obvious whether differences in the observed counts
  and the expected counts from the geometric distribution are
  significantly different.
  That is, it is not clear whether these deviations might be
  due to chance or whether they are so strong that the data
  provide convincing evidence against the null hypothesis.
  However, we can perform a chi-square test using the counts
  in Figure~\ref{sAndP500TimeToPosTrade2}.
\end{nexample}
\end{examplewrap}

\begin{exercisewrap}
\begin{nexercise}
Figure~\ref{sAndP500TimeToPosTrade2}
provides a set of count data for waiting times
($O_1=\spdaysA{}$, $O_2=\spdaysB{}$, ...)
and expected counts under the geometric distribution
($E_1=\spdaysEA{}$, $E_2=\spdaysEB{}$, ...).
Compute the chi-square test statistic, $X^2$.\footnotemark
\end{nexercise}
\end{exercisewrap}
\footnotetext{$X^2 =
      \frac{(\spdaysA{}-\spdaysEA{})^2}{\spdaysEA{}} +
      \frac{(\spdaysB{}-\spdaysEB{})^2}{\spdaysEB{}} +
      \cdots +
      \frac{(\spdaysG{}-\spdaysEG{})^2}{\spdaysEG{}}
    = \spdaysXSq{}$}

\begin{exercisewrap}
\begin{nexercise}
Because the expected counts are all at least~5,
we can safely apply the chi-square distribution to $X^2$.
However, how many degrees of freedom should
we~use?\footnotemark{}
\end{nexercise}
\end{exercisewrap}
\footnotetext{There are $k = \spdaysN{}$ groups, so we use
    $df = k - 1 = \spdaysDF{}$.}

\begin{examplewrap}
\begin{nexample}{If the observed counts follow the
    geometric model, then the chi-square test statistic
    $X^2 = \spdaysXSq{}$ would closely follow a chi-square
    distribution with $df = \spdaysDF{}$.
    Using this information, compute a p-value.} 
  \label{DNRejectGeomModelForSP500}%
  Figure~\ref{geomFitPValueForSP500} shows the
  chi-square distribution, cutoff, and the shaded p-value.
  % We could look up $X^2 = \spdaysXSq{}$ in
  % Appendix~\ref{chiSquareProbabilityTable} to determine
  % that the p-value is greater than 0.3.
  Using software, we can find the p-value: \spdaysPvalue{}.
  Ultimately, we do not have sufficient evidence to reject
  the notion that the wait times follow a geometric
  distribution for the last \spyears{} years of data
  for the S\&P500,
  i.e. we cannot reject the notion that trading days
  are independent.
\end{nexample}
\end{examplewrap}

\begin{figure}[h]
  \centering
  \Figure{0.9}{geomFitPValueForSP500}
  \caption{Chi-square distribution with \spdaysDF{}
      degrees of freedom.
      The p-value for the stock analysis is shaded.}
  \label{geomFitPValueForSP500}
\end{figure}

\begin{examplewrap}
\begin{nexample}{In
    Example~\ref{DNRejectGeomModelForSP500},
    we did not reject the null hypothesis that the trading days
    are independent during the last \spyears{} of data.
    Why is this so important?}
  It may be tempting to think the market is ``due'' for
  an \resp{Up} day if there have been several consecutive
  days where it has been down.
  However, we haven't found strong evidence that there's
  any such property where the market is ``due'' for
  a correction.
  At the very least, the analysis suggests any dependence
  between days is very weak.
\end{nexample}
\end{examplewrap}

\index{data!S\&P500 stock data|)}

\CalculatorVideos{the chi-square goodness of fit test}



%__________________
\section{Testing for independence in two-way tables}
\label{twoWayTablesAndChiSquare}

\index{data!iPod|(}

\newcommand{\iPodAA}{2}
\newcommand{\iPodAB}{23}
\newcommand{\iPodAC}{36}
\newcommand{\iPodAD}{61}
\newcommand{\iPodAFraction}{0.2785}
\newcommand{\iPodAExpected}{20.33}
\newcommand{\iPodBA}{71}
\newcommand{\iPodBB}{50}
\newcommand{\iPodBC}{37}
\newcommand{\iPodBD}{158}
\newcommand{\iPodBFraction}{0.7215}
\newcommand{\iPodBExpected}{52.67}
\newcommand{\iPodDA}{73}
\newcommand{\iPodDB}{73}
\newcommand{\iPodDC}{73}
\newcommand{\iPodDD}{219}
\newcommand{\iPodN}{\iPodDD}

We all buy used products --
cars, computers, textbooks, and so on --
and we sometimes assume the sellers of those products
will be forthright about any underlying problems with
what they're selling.
This is not something we should take for granted.
Researchers recruited \iPodN{} participants in a study where they
would sell a used iPod\footnote{For readers not as old as
  the authors, an iPod is basically an iPhone without
  any cellular service, assuming it was one of the later
  generations. Earlier generations were more basic.}
that was known to have frozen twice in the past.
The participants were incentivized to get as much money
as they could for the iPod since they would receive a 5\%
cut of the sale on top of \$10 for participating.
The researchers wanted to understand what types of questions
would elicit the seller to disclose the freezing issue.

Unbeknownst to the participants who were the sellers
in the study,
the buyers were collaborating with the researchers
to evaluate the influence of different questions
on the likelihood of getting the sellers to disclose
the past issues with the iPod.
The scripted buyers started with
``Okay, I guess I'm supposed to go first.
  So you've had the iPod for 2 years ...''
and ended with one of three questions:
\begin{itemize}
\item General: What can you tell me about it?
\item Positive Assumption: It doesn't have any problems, does it?
\item Negative Assumption: What problems does it have?
\end{itemize}
The question is the treatment given to the sellers,
and the response is whether the question prompted them
to disclose the freezing issue with the iPod.
The results are shown in Figure~\ref{ipod_ask_data_summary},
and the data suggest that asking the,
\emph{What problems does it have?},
was the most effective at getting the seller to disclose
the past freezing issues.
However, you should also be asking yourself:
could we see these results due to chance alone,
or is this in fact evidence that some questions
are more effective for getting at the truth?

\begin{figure}[ht]
\centering
\begin{tabular}{l ccc l}
  \hline
  & General & Positive Assumption &
      Negative Assumption & Total \\ 
  \hline
  Disclose Problem & \iPodAA{} &  \iPodAB{} &
      \iPodAC{} & \iPodAD{} \\ 
  Hide Problem &  \iPodBA{} &  \iPodBB{} &
      \iPodBC{} & \iPodBD{} \\ 
  \hline
  Total & \iPodDA{} & \iPodDB{} &
      \iPodDC{} & \iPodDD{} \\
  \hline
\end{tabular}
\caption{Summary of the iPod study, where a question was
  posed to the study participant who acted}
\label{ipod_ask_data_summary}
\end{figure}

\begin{onebox}{Differences of one-way tables vs two-way tables}
  A one-way table describes counts for each outcome in a single
  variable.
  A two-way table describes counts for \emph{combinations}
  of outcomes for two variables.
  When we consider a two-way table, we often would like to know,
  are these variables related in any way?
  That is, are they dependent (versus independent)?
\end{onebox}

The hypothesis test for the iPod experiment is really about
assessing whether there is statistically significant evidence
that the success each question had on getting the participant
to disclose the problem with the iPod.
In other words, the goal is to check whether the buyer's
question was independent of whether the seller disclosed
a problem.


\D{\newpage}

\subsection{Expected counts in two-way tables}

\noindent%
Like with one-way tables, we will need to compute
estimated counts for each cell in a two-way table.

\begin{examplewrap}
\begin{nexample}{From the experiment,
    we can compute the proportion of all sellers who disclosed
    the freezing problem as $\iPodAD{}/\iPodDD = \iPodAFraction{}$.
    If there really is no difference among the questions
    and 27.85\% of sellers were going to disclose the freezing
    problem no matter the question that was put to them,
    how many of the \iPodDA{} people in the \resp{General}
    group would we have expected to disclose the freezing
    problem?} \label{iPodExComputeExpAA}
  We would predict that $\iPodAFraction{} \times \iPodDA{} = \iPodAExpected{}$
  sellers would disclose the problem.
  Obviously we observed fewer than this, though it is not
  yet clear if that is due to chance variation or whether
  that is because the questions vary in how effective they
  are at getting to the truth.
\end{nexample}
\end{examplewrap}

\begin{exercisewrap}
\begin{nexercise}\label{iPodExComputeExpBB}
If the questions were actually equally effective,
meaning about 27.85\% of respondents would disclose the
freezing issue regardless of what question they were asked,
about how many sellers would we expect to \emph{hide} the
freezing problem from the Positive Assumption
group?\footnotemark
\end{nexercise}
\end{exercisewrap}
\footnotetext{We would expect
    $(1 - \iPodAFraction{}) \times \iPodDA{} = \iPodBExpected{}$.
    It is okay that this result,
    like the result from Example~\ref{iPodExComputeExpAA},
    is a fraction.}

We can compute the expected number of sellers who we would
expect to disclose or hide the freezing issue for all groups,
if the questions had no impact on what they disclosed,
using the same strategy employed in
Example~\ref{iPodExComputeExpAA} and
Guided Practice~\ref{iPodExComputeExpBB}.
These expected counts were used to construct Figure~\ref{ipod_ask_data_summary_expected},
which is the same as Figure~\ref{ipod_ask_data_summary},
except now the expected counts have been added in parentheses.

\begin{figure}[h]
\centering
\begin{tabular}{l lll l}
  \hline
  & General & Positive Assumption &
      Negative Assumption & Total \\ 
  \hline
  Disclose Problem &
      \iPodAA{} \ \highlightO{\footnotesize(\iPodAExpected{})} &
      \iPodAB{} \highlightO{\footnotesize(\iPodAExpected{})} &
      \iPodAC{} \highlightO{\footnotesize(\iPodAExpected{})} &
      \iPodAD{} \\ 
  Hide Problem &
      \iPodBA{} \highlightO{\footnotesize(\iPodBExpected{})} &
      \iPodBB{} \highlightO{\footnotesize(\iPodBExpected{})} &
      \iPodBC{} \highlightO{\footnotesize(\iPodBExpected{})} &
      \iPodBD{} \\ 
  \hline
  Total & \iPodDA{} & \iPodDB{} &
      \iPodDC{} & \iPodDD{} \\
  \hline
\end{tabular}
\caption{The observed counts and the
    \highlightO{(expected counts)}.}
\label{ipod_ask_data_summary_expected}
\end{figure}

The examples and exercises above provided some help
in computing expected counts.
In general, expected counts for a two-way table may
be computed using the row totals, column totals,
and the table total.
For instance, if there was no difference between the groups,
then about 27.85\% of each column should be in the first row:
\begin{align*}
\iPodAFraction{}\times (\text{column 1 total}) &= \iPodAExpected{} \\
\iPodAFraction{}\times (\text{column 2 total}) &= \iPodAExpected{} \\
\iPodAFraction{}\times (\text{column 3 total}) &= \iPodAExpected{}
\end{align*}
Looking back to how \iPodAFraction{} was computed --
as the fraction of sellers who disclosed the freezing issue
($\iPodBD{}/\iPodDD{}$) --
these three expected counts could have been computed as
\begin{align*}
\left(\frac{\text{row 1 total}}{\text{table total}}\right)
    \text{(column 1 total)} &= \iPodAExpected{} \\
\left(\frac{\text{row 1 total}}{\text{table total}}\right)
    \text{(column 2 total)} &= \iPodAExpected{} \\
\left(\frac{\text{row 1 total}}{\text{table total}}\right)
    \text{(column 3 total)} &= \iPodAExpected{}
\end{align*}
This leads us to a general formula for computing expected
counts in a two-way table when we would like to test whether
there is strong evidence of an association between the column
variable and row variable.

\begin{onebox}{Computing expected counts in a two-way table}
  To identify the expected count for the $i^{th}$ row
  and $j^{th}$ column, compute
  \begin{align*}
  \text{Expected Count}_{\text{row }i,\text{ col }j}
    = \frac{(\text{row $i$ total}) \times
        (\text{column $j$ total})}{\text{table total}}\vspace{2mm}
  \end{align*}
\end{onebox}


\subsection{The chi-square test for two-way tables}

The chi-square test statistic for a two-way table is found
the same way it is found for a one-way table.
For each table count, compute
\begin{align*}
&\text{General formula} &&
    \frac{(\text{observed count } - \text{expected count})^2}
        {\text{expected count}} \\
&\text{Row 1, Col 1} &&
    \frac{(\iPodAA - \iPodAExpected)^2}{\iPodAExpected} = 16.53 \\
&\text{Row 1, Col 2} &&
    \frac{(\iPodAB - \iPodAExpected)^2}{\iPodAExpected} = 0.35 \\
& \hspace{9mm}\vdots &&
    \hspace{13mm}\vdots \\
&\text{Row 2, Col 3} &&
    \frac{(\iPodBC - \iPodBExpected)^2}{\iPodBExpected} = 4.66
\end{align*}
Adding the computed value for each cell gives the chi-square test statistic $X^2$:
\begin{align*}
X^2 = 16.53 + 0.35 + \dots + 4.66 = 40.13
\end{align*}
Just like before, this test statistic follows a chi-square distribution. However, the degrees of freedom are computed a little differently for a two-way table.\footnote{Recall: in the one-way table, the degrees of freedom was the number of cells minus 1.} For two way tables, the degrees of freedom is equal to
\begin{align*}
df = \text{(number of rows minus 1)}\times \text{(number of columns minus 1)}
\end{align*}
In our example, the degrees of freedom parameter is
\begin{align*}
df = (2-1)\times (3-1) = 2
\end{align*}
If the null hypothesis is true
(i.e. the questions had no impact on the sellers in
    the experiment),
then the test statistic $X^2 = 40.13$ closely follows
a chi-square distribution with 2 degrees of freedom.
Using this information, we can compute the p-value for
the test, which is depicted in
Figure~\ref{iPodChiSqTail}.

\begin{onebox}{Computing degrees of freedom for a two-way table}
  When applying the chi-square test to a two-way table,
  we use
  \begin{align*}
  df = (R-1)\times (C-1)
  \end{align*}
  where $R$ is the number of rows in the table
  and $C$ is the number of columns.
\end{onebox}

When analyzing 2-by-2 contingency tables, one guideline
is to use the two-proportion methods introduced in
Section~\ref{differenceOfTwoProportions}.

\begin{figure}[h]
\centering
\includegraphics[width=0.65\textwidth]{ch_inference_for_props/figures/iPodChiSqTail/iPodChiSqTail}
\caption{Visualization of the p-value for $X^2 = 40.13$
    when $df = 2$.}
\label{iPodChiSqTail}
\end{figure}

\begin{examplewrap}
\begin{nexample}{Find the p-value and draw a conclusion
    about whether the question affects the sellers likelihood
    of reporting the freezing problem.}
  % Looking in Appendix~\ref{chiSquareProbabilityTable}
  % on page~\pageref{chiSquareProbabilityTable},
  % we examine the row corresponding to 2 degrees of freedom.
  % The test statistic, $X^2 = 40.13$,
  % is larger than the value in the last column,
  % meaning the tail area and p-value are smaller than 0.001.
  Using a computer, we can compute a very precise value
  for the tail area above $X^2 = 40.13$ for a chi-square
  distribution with 2 degrees of freedom:
  0.000000002.
  (If using the table in
    Appendix~\ref{chiSquareProbabilityTable},
    we would identify the p-value is smaller
    than 0.001.)
  Using a significance level of $\alpha=0.05$,
  the null hypothesis is rejected since the p-value is smaller.
  That is, the data provide convincing evidence that the
  question asked did affect a seller's likelihood to tell
  the truth about problems with the iPod.
\end{nexample}
\end{examplewrap}

\index{data!iPod|)}

\index{data!diabetes|(}

\begin{examplewrap}
\begin{nexample}{Figure~\ref{diabetes2ExpMetRosiLifestyleSummary}
    summarizes the results of an experiment evaluating
    three treatments for Type~2 Diabetes in patients
    aged 10-17 who were being treated with metformin.
    The three treatments considered were
    continued treatment with metformin (\resp{met}),
    treatment with metformin combined with rosiglitazone
    (\resp{rosi}),
    or a lifestyle intervention program.
    Each patient had a primary outcome, which was either lacked
    glycemic control (failure)
    or did not lack that control (success).
    What are appropriate hypotheses for this test?}
  \label{diabetes2ExpMetRosiLifestyleIntroExample}
  \begin{itemize}
  \item[$H_0$:] There is no difference in the effectiveness
      of the three treatments.
  \item[$H_A$:] There is some difference in effectiveness
      between the three treatments, e.g. perhaps the
      \resp{rosi} treatment performed better than
      \resp{lifestyle}.
  \end{itemize}
\end{nexample}
\end{examplewrap}

\begin{figure}[h]
\centering
\begin{tabular}{l ccc l}
\hline
 & Failure & Success & Total \\ 
\hline
\resp{lifestyle} & 109 & 125 & 234 \\ 
\resp{met} & 120 & 112 & 232 \\ 
\resp{rosi} &  90 & 143 & 233 \\ 
\hline
Total & 319 & 380 & 699 \\
\hline
\end{tabular}
\caption{Results for the Type~2 Diabetes study.}
\label{diabetes2ExpMetRosiLifestyleSummary}
\end{figure}

\D{\newpage}

\begin{exercisewrap}
\begin{nexercise}
A chi-square test for a two-way table may be used to test
the hypotheses in
Example~\ref{diabetes2ExpMetRosiLifestyleIntroExample}.
As a first step, compute the expected values for each of the
six table cells.\footnotemark{}
\end{nexercise}
\end{exercisewrap}
\footnotetext{The expected count for
    row one / column one is found by multiplying the
    row one total (234) and column one total (319),
    then dividing by the table total (699):
    $\frac{234\times 319}{699} = 106.8$.
    Similarly for the second column and the first row:
    $\frac{234\times 380}{699} = 127.2$.
    Row 2: 105.9 and 126.1.
    Row 3: 106.3 and 126.7.}

\begin{exercisewrap}
\begin{nexercise}
Compute the chi-square test statistic for the data in
Figure~\ref{diabetes2ExpMetRosiLifestyleSummary}.\footnotemark
\end{nexercise}
\end{exercisewrap}
\footnotetext{For each cell,
    compute $\frac{(\text{obs} - \text{exp})^2}{exp}$.
    For instance, the first row and first column:
    $\frac{(109-106.8)^2}{106.8} = 0.05$.
    Adding the results of each cell gives the
    chi-square test statistic:
    {\scriptsize$X^2 = 0.05 + \cdots + 2.11 = 8.16$}.}

\begin{exercisewrap}
\begin{nexercise}
Because there are 3 rows and 2 columns,
the degrees of freedom for the test is
$df = (3 - 1) \times (2 - 1) = 2$.
Use $X^2 = 8.16$, $df = 2$, evaluate whether
to reject the null hypothesis using a significance level
of~0.05.\footnotemark
\end{nexercise}
\end{exercisewrap}
\footnotetext{
    If using a computer, we can identify the p-value
    as 0.017.
    That is, we reject the null hypothesis because
    the p-value is less than 0.05, and we conclude
    that at least one of the treatments is more or
    less effective than the others at treating
    Type~2 Diabetes for glycemic control.}

\index{data!diabetes|)}

\CalculatorVideos{the chi-square test for independence}
